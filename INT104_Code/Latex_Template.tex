\documentclass[10pt,twocolumn]{article}

\usepackage[utf8]{inputenc}
% \documentclass{article}
\usepackage{layout}


\usepackage[left=2cm,right=2cm,top=2.5cm,bottom=2.5cm]{geometry}
\setlength{\columnsep}{0.2in}
\usepackage{graphicx}

\renewcommand{\baselinestretch}{1}

\begin{document}
\begin{titlepage}
    \begin{center}
    % 	HEADING & LOGO
        \begin{figure}[h]
	        \centering
	        \includegraphics[scale=1]{XJTLU.png}
        \end{figure}
        \textsc{
        % \\[1cm]代表中间空1cm高度
        \\[2.5cm]
        % \\Huge代表临时改变字号大小,以及Large, LARGE, HUGE
            \Huge{INT104 Coursework 2 Report Guidance}\\[2.5cm]
            \LARGE
        %  Author Name
            Sample Student (1234567)
        \\[1cm]
        Lab-A-Group-1
    \\[1cm]
    {\today}\\[.5cm]
}
    \end{center}
\end{titlepage}

\section{Introduction}
This is the template of INT104 Coursework 2 report. You have to follow the requirement on LMO to complete the tasks. \textbf{You have to complete the report in 8 pages}, exclude abstract, reference, and appendix. 

\subsection{Page Format}
The report should have a cover, which shows your project title, name, ID number, Group Number, and date. The recommended font is \textit{Times New Roman}, 10-11pt. The recommended page format is two columns, 2cm margins for left and right edges, 2.54cm margins for top and bottom edges, 1-1.25 line space, with page number at the bottom. \textbf{It is OK to use your own format if you think yours looks better.}

\subsection{Reference Format}
\textbf{This report does not limit the format of reference}. \textit{IEEE}, \textit{APA}, or \textit{Harvard} are all acceptable. Just make sure you have the correct citation format.

\subsection{Introduction Part}
In introduction or abstract part, you may have the following content:
\begin{enumerate}
    \item Briefly describe the requirement of this coursework
    \item What classifiers you have built
    \item What is the result.
\end{enumerate}

You \textbf{don't} have to write too much background knowledge. Just include the necessary background and literature review, depends on you.


\section{Dimensionality reduction}
Section 2, 3, and 4 refers to the 3 tasks in this coursework. \textbf{You need to avoid putting screenshots of your code directly in the report}. Instead, clear plots made by \textit{matplotlib} or \textit{Excel} would be better choices.
Your report should include, but not limited to the following content:
\begin{enumerate}
    \item Please generally introduce the dimensionality reduction algorithm you chosen, such as Principal Component Analysis (PCA). And how the number of feature dimension decided? 
    \item With the presented data, the cumulative sum of explained variance ratio of each principal component could be presented. Data visualization will be recommended.
\end{enumerate}

\section{Training Classifiers in a Supervised Way}
For classifiers, it is recommended to do multiple experiments to compare the difference, for example: different input data for one classifier, changing the variables for a classifier, or the same data for different classifiers. \textbf{An evaluation part is necessary for each classifier.}
You may answer the following questions:

\begin{enumerate}
    \item Which classifiers (no more than five, no less than three) have you tried in this task? Please generally introduce the classifiers you used.
    \item With knowledge obtained in the previous task, in which way your data is pre-processed as the input of the three classifiers?
    \item How are the classifiers trained? How are the classifiers used for inference processes? 
    \item How do you perform cross validation for the classifiers? What the results of the resulting classifiers are? Give the evaluation for your classifiers.
\end{enumerate}

\section{Unsupervised Classification}
You may answer the following questions.

\begin{enumerate}
    \item Select ONE unsupervised classifier to try. Please generally introduce the classifier you used.
    \item Please give the full details of the unsupervised classification processes.
    \item Attempt to interpret how the students are grouped according to the resulting classifier.
\end{enumerate}

\section{Conclusion}
\textbf{Clear results and independent point of view are the key to getting a high score}.

\end{document}